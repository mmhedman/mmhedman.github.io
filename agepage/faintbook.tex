Observations between Voyager and Cassini


Faint rings in and around the Main rings. 

Nestled among and around Saturn's main rings are multiple low optical depth rings composed primarily of dust grains. These features therefore have similar composition as the more extensive dusty rings found further away from Saturn, but are in very different dynamical and plasma environments due to the proximity of the Saturn and the high optical depth rings. 

Dusty ringlets occur within all gaps in the main rings more than approximately 50 kilometers across. Narrower gaps seem to be devoid of discrete ringlets, although we cannot rule out a constant density dust sheets within these gaps. The most well known such ringlets occupy the Encke Gap along with the moon Pan. Strangely, this gap is the only one that contains three ringlets and have significant brightness variations with longitude. The remaining gaps contain single ringlets. 

Most of these features have a finite eccentricity. For the ringlets in the Encke Gap and the outer Cassini division also exhibit heliotropic behavior, with the ringlet furthest  from the planet at a longitude closest to the sun.

Inside the C ring lies the dusty D ring, the innermost component of the 