Review of "The Age of Everything" in 
Ciel et Terre (Bulletin of the Belgian Royal Society of Astronomy, Meteorology and Earth Science)
Volume 124, 1 January/Februray 2008
By R. DEJAIFEE

(Translated from the French)

A surprising book in the way that it shows an eclectic character in its twelve chapters respectively devoted to: (1) a brief general introduction; (2) the calendars of the [ancient] Mayans; (3) the phenomenon of precession, the pole star and the age of the pyramids; (4) the physics of carbon-14; (5) the problem of the standardization of the dates from the carbon-14 technique and the history of the air; (6) the problems of carbon-14 and the peopling of the New World; (7) the dating techniques of potassium/argon and DNA and the age of fossil hominids; (8) the puzzle of molecular dating and the numerous different types of mammals; (9) meteorites and the age of the solar system; (10) colors, luminosity and the age of stars; (11) distances, redshifts and the age of the universe, (12) the parameterization of the age of the universe.

In fact, the author, a researcher in the astronomy department at Cornell University, intends to consider the problem of different ages --from of order a thousand years (Mayans) up to 14 billion years (universe)-- and more precisely that of the determination of a very vast range of ages, through a real dissection of diverse methods, (which he) uses, in an interdisciplinary manner, to try to explain how archaeologists, biologists, geologists, physicists, astronomers and cosmologists each reconstruct a more or less distant past. Above all, the author seeks in this way to explain how scientific research
has been able  to determine different matters of orders of magnitude with the aid of dating techniques capable of dating climatic changes distant or the models of human migrations, as well as the age of stars and the universe. In short, the author intends to cover a wide range of  time scales. At once a sort of challenge and a view ahead or far back that demonstrates the real capacity of modern science to enable us to make contact with the distant past. Such a book will be be interesting, even useful, for all people who have an marked interest in questions of traveling or harking back through  time. 

Certain subjects apparently(?), such as the techniques of the standardization, i.e. carbon-14 dating employs a phase to obtain the raw data after which  the dates must be properly "calibrated",  were introduced to  the author from the start of his university studies and since that time one (he?) regretted the literary, incorrect popular manner, i.e.  the general presentation to the larger public of the "forks" or uncertainties associated with several of these dating methods.

A good idea and a good theme, covering perhaps a subject too broad to approach exclusively in this manner. There results from it a certain mixture of genres and a real difficulty in comprehending certain parts of the text that require a careful reading --even sometimes a second reading-- before they reveal  the profound meaning or the validity of the given explanations. In short, a book that attempts to make comprehensible how diverse types of very different researchers, each engaged in one small part of a great variety of scientific topics, determine the age of events that they study, and that came from a series of talks or expositions given by the author while at the Institute for Cosmological Physics at the University of Chicago. The lists of suggested readings on each of the discussed themes is more useful for those readers who are particularly interested

