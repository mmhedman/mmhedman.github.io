<i> Added 2/1/09 </i> In the second (2008) edition of "Chronicles of the Mayan Kings and Queens" by S. Martin and N. Grube, the authors report on a revised interpretation of of one of the events involving Yuknoom Ch'een, B'alah Chan K'awiil and Nuun Ujol Chaak. Specifically, they now state that Nuun Ujol Chaak of Tikal did not arrive in Palenque in 659 after his defeat by Calakmul. This revision is based on work by D. Stuart referenced obliquely in a chapter by S. Martin in "Tikal: Dynasties, Foreigners and Affairs of State" (2003, edited by J.A. Sabloff), which asserts that the character named "Nuun Ujol Chaak" in the Palenque inscription is denoted as the ruler of a city called the "Wa-bird polity" (located in Santa Elena, not far from Palenque) instead of Mutual/Tikal. These authors therefore argue the king mentioned in the Palenque inscription is a different king who just happened to have the same name as the ill-fated king of Tikal.
<br>
<br>
While I must defer to the experts regarding the reading of what city the Nuun Ujol Chaak in the Palenque inscription was supposed to have ruled over, I am not entirely sure I accept the idea that there just happened to be two kings named "Nuun Ujol Chaak" ruling two different cities at the same time. As Martin points out in his chapter on Tikal, the date in the Palenque inscription is exactly 1 k'atun from B'alah Chan K'awiil's final defeat of Nuun Ujol Chaak 9.12.6.16.17 (11 Kaban 10 Sotz, note this event was not included in the appendix to Chapter 2 because the texts did not reference Yuknoom Ch'een directly). This seems to be a bit of a coincidence, and hints that the authors of either the Palenque or the Dos Pilas texts were trying to tie these two events together. This (along with the fact that Calakmul operating in nearby cities like Moral and Piedras Negras at this time) leads me to wonder if the Nuun Ujol Chaak of Santa Elena really was the same as the Nuun Ujol Chaak of Tikal. Perhaps Nuun Ujol Chaak chose Santa Elena as a base for operations when he could not rule in Tikal. Obviously, this is merely wild speculation on my part, and it is indeed possible that these coincidences in names and dates are just coincidences. Regardless, it would be very interesting to see if monuments from Santa Elena dating from this time period could shed light on this situation.
 