Did the Ancient Mayans predict the world is going to end in 2012? Almost certainly not. No Ancient Mayan text (at least none I am aware of) states explicitly that the world is going to end in December 2012. Instead, the notion that the Ancient Mayans "predicted" that something significant would happen in late 2012 is based on the fact that this date in the Mayan Calendar is a bit like our year 2000, it is a nice round number. As such, it marks a significant milestone within the Mayan Calendar, but there is no reason to assume this has dire implications. In fact, certain Mayan texts suggest that the Ancient Mayans conceived of time-spans much greater than our current estimates of the age of the entire universe.

It is clear that the Ancient Mayans had a special interest in the passage of time. During the so-called Classic Period (roughly 1000 to 1800 years ago), the Mayan produced numerous stone inscriptions in a region of Central America that now includes Southeastern Mexico, Belize, Guatemala and Western Honduras. Many of these inscriptions provide dates for various events in one of the most sophisticated calendars every devised. One component of this calendar is the so-called "Long Count", a number that specifies how many days have passed since one particular day deep in the past. Using data from shortly after the European conquistadors first arrived in Mesoamerica (when portions of the Mayan calendar were still in use) modern scholars have determined that the "day 0" of the Long Count corresponds to some day in August 3114 B.C.E. (the exact day is still somewhat in doubt). This is thousands of years before the earliest known dated inscriptions in Central America,  so most of the known Long Counts, which record events roughly contemporary with the inscription itself, are extremely big numbers. For example, one text records that a woman was born 1,383,136 days after the nominal start of the Long Count.

Within this count of days, certain intervals of time were given special attention. These intervals corresponded to round numbers in their counting system, and thus are something like decades (10 years) or centuries (100 years) in our culture, which we often use to mark major milestones in our history.  However, while our modern numbering system is based on powers of 10,  the Mayans used a numbering system that was based on powers of 20. Thus, for the Mayans a significant period of time was not so much a decade of 10 years, but instead something called the K'atun, which lasted nearly 20 years (technically 20*360=7200 days). The start of a K'atun was often considered a good time for the building of various monuments, and many dates contain references to various K'atun periods. 

In addition to the K'atun, there were several other cycles of time that the Mayans used in their calendar. The one that is relevant here consisted or 260 K'atuns (1,872,000 days or about 5125 years). Why the cycle of 260 K'atuns was considered particularly interesting to the Mayans is still a subject of research, but it may have something to do with a cycle of 260 days that was another important part of the Mayan Calendar. The current cycle of 260 K'atuns started in August 3114 B.C (the "day 0" of the Long Count), so only a little math is needed to show that in December 2012, this cycle of K'atuns will come to an end. However, there is no reason that just because the a cycle of 260 K'atuns was coming to an end means that some disaster will befall the world. Indeed, looking at certain inscriptions show that the Mayans themselves did not consider the cycle of 260 K'atuns as the end-all and be-all of existence.

For instance, in the Mayan city of Palenque, there is a cluster of three temples with inscriptions commissioned by a Ruler known an K'inich' Kan Bahlam ("Radiant Snake Jaguar"). These inscriptions describe a series of events that occurred around the start of the 260-K'atun cycle. Such ancient events are probably best understood as mythological. Even so, the earliest of these events (recorded in the so-called "Temple of the Cross") was the birth of some person or diety several years before the start of the current 260-K'atun cycle. The actions of this and other characters continue into the current 260-K'atun cycle without obvious interruption, suggesting that while the Mayans though the start of the cycle was worth noting, they did not necessarily think it involved a major cataclysm. 

In fact, the Mayans conceived of much longer time-scales than the 260-K'atun cycle. One text on a hieroglyphic Stairway (Number 2, Step 7) in Yaxchilan records a period of time amounting to over 4 trillion  (4*10^12) years. Another text, known as Coba stela 1, mentions a time period more than 40,000,000,000,000,000,000,000,000,000 years long!  Clearly, the Mayans had no trouble thinking that the world had been in existence for a very long time, and, we might presume that they expected it to last an equally long time into the future.

Given all this, it seems that instead of expecting impeding doom in December 2012, we should instead prepare to celebrate this occasion and consider the span of time that both separates us from the Ancient Mayans and  connects us to them. 

Further Reading:

To learn more about Mayan calendars, see such books as "Reading the Mayan Glyphs" by Coe and Stone and "How to Read Maya Hieroglyphs" by Montogomery.

For translations of the Inscriptions at Palenque, see "Understanding Maya Inscriptions" by Harris and Stearns.

For translations of the inscription from Yaxchilan, see "Yaxchilan: The design of a Maya ceremonial center" by Tate.

For the Coba Stela, see "How to Read Maya Hieroglyphs" by Montgomery 

